%*********************************************************************************
% impostazioni-articolo.tex
% di Lorenzo Pantieri (2012)
% file che contiene le impostazioni dell'articolo
%*********************************************************************************


%*********************************************************************************
% Comandi personali
%*******************************************************
\newcommand{\myName}{Iuri La Rosa, Federico Muciaccia}                            % autore
\newcommand{\myTitle}{Antenna network breakdown}  % titolo
\date{}                                                           % nessuna data

\title{\myTitle}
\author{\myName}



%*********************************************************************************
% Impostazioni di amsmath, amssymb, amsthm
%*********************************************************************************

% comandi per gli insiemi numerici (serve il pacchetto amssymb)
\newcommand{\numberset}{\mathbb} 
\newcommand{\N}{\numberset{N}} 
\newcommand{\R}{\numberset{R}} 

% un ambiente per i sistemi
\newenvironment{sistema}%
  {\left\lbrace\begin{array}{@{}l@{}}}%
  {\end{array}\right.}

% definizioni (serve il pacchetto amsthm)
\theoremstyle{definition} 
\newtheorem{definizione}{Definizione}

% teoremi, leggi e decreti (serve il pacchetto amsthm)
\theoremstyle{plain} 
\newtheorem{teorema}{Teorema}
\newtheorem{legge}{Legge}
\newtheorem{decreto}[legge]{Decreto}
\newtheorem{murphy}{Murphy}[section]



%*********************************************************************************
% Impostazioni di biblatex
%*********************************************************************************
\defbibheading{bibliography}{%
\phantomsection 
\addcontentsline{toc}{section}{\refname}
\section*{\bibname\markboth{\MakeUppercase{\refname}}{\MakeUppercase{\refname}}}}



%*********************************************************************************
% Impostazioni di listings
%*********************************************************************************
\addto\captionsitalian{%
\renewcommand{\lstlistingname}{Snippet}}

\addto\captionsitalian{%
\renewcommand{\lstlistlistingname}{Elenco degli snippet}}

\lstset{language=[GNU]C++,
	basicstyle=\small\ttfamily,
	keywordstyle=\color{Black}\bfseries,
	identifierstyle=\color{Black},
	commentstyle=\color{DarkGreen},
	stringstyle=\color{Red}\ttfamily,
	numbers=none,
	numberstyle=\scriptsize,
	tabsize=4,
	stepnumber=1,
	numbersep=8pt,
	showstringspaces=false,
	breaklines=true,
	frame=single
%     emphstyle=\color{Goldenrod}
} 



%*********************************************************************************
% Impostazioni di hyperref
%*********************************************************************************
\hypersetup{%
    hyperfootnotes=false,pdfpagelabels,
    %draft,	% = elimina tutti i link (utile per stampe in bianco e nero)
    colorlinks=true, linktocpage=true, pdfstartpage=1, pdfstartview=FitV,%
    % decommenta la riga seguente per avere link in nero (per esempio per la stampa in bianco e nero)
    %colorlinks=false, linktocpage=false, pdfborder={0 0 0}, pdfstartpage=1, pdfstartview=FitV,% 
    breaklinks=true, pdfpagemode=UseNone, pageanchor=true, pdfpagemode=UseOutlines,%
    plainpages=false, bookmarksnumbered, bookmarksopen=true, bookmarksopenlevel=1,%
    hypertexnames=true, pdfhighlight=/O,%nesting=true,%frenchlinks,%
    urlcolor=webbrown, linkcolor=RoyalBlue, citecolor=webgreen, %pagecolor=RoyalBlue,%
    %urlcolor=Black, linkcolor=Black, citecolor=Black, %pagecolor=Black,%
    pdftitle={\myTitle},%
    pdfauthor={\textcopyright\ \myName},%
    pdfsubject={},%
    pdfkeywords={},%
    pdfcreator={pdfLaTeX},%
    pdfproducer={LaTeX with hyperref and ClassicThesis}%
}



%*********************************************************************************
% Impostazioni di graphicx
%*********************************************************************************
\graphicspath{{Immagini/}} % cartella dove sono riposte le immagini



%*********************************************************************************
% Impostazioni di xcolor
%*********************************************************************************
\definecolor{webgreen}{rgb}{0,.5,0}
\definecolor{webbrown}{rgb}{.6,0,0}



%*********************************************************************************
% Impostazioni di caption
%*********************************************************************************
\captionsetup{tableposition=top,figureposition=bottom,font=small,format=hang,labelfont=bf}



%*********************************************************************************
% Altro
%*********************************************************************************

% [...] ;-)
\newcommand{\omissis}{[\dots\negthinspace]}

% eccezioni all'algoritmo di sillabazione
\hyphenation{Fortran ma-cro-istru-zio-ne nitro-idrossil-amminico}
%aumentato il carattere a 12 punti (prima era i 11)
\documentclass[a4paper,titlepage,twoside,openright]{book}
\usepackage[italian]{babel}
\usepackage[utf8]{inputenc}
\usepackage{pslatex}
\usepackage{epsfig}
%aggiunti per funzionare con eepic
%\usepackage{eepic,epic,eepicemu}
\usepackage{color}
% per grafici in latex come sopra
%\usepackage{longtable}
\usepackage{graphicx}
\usepackage{latexsym}
\usepackage{epstopdf}
\usepackage{microtype}%migliora la resa dei caratteri
% \usepackage[pdftex]{hyperref}
\usepackage{booktabs}
\usepackage{caption}
\usepackage{listings}
\usepackage{subfig}
\usepackage{wrapfig}
\usepackage{multirow}
\usepackage{amsmath,amssymb}
\usepackage{textcomp}
\usepackage[]{frontespizio}

\begin{document}

\begin{frontespizio}
	\Istituzione{ }
	\Logo{../Immagini/Sigillo}
	\Facolta{Scienze MM. FF. NN.}
	\Corso[Laurea Magistrale]{Fisica}
	\Annoaccademico{2014-2015}
	\Titoletto{Relazione per il corso di Laboratorio di Fisica Nucleare e Subnucleare}
	\Titolo{Misure di radioattività e fluorescenza X\\ con un rivelatore al Germanio ultrapuro}
% 	\Sottotitolo{Un'analisi generale sulle stelle di materia degenere}
	\Candidato{Valentina Dompè}
	\Candidato{Iuri La Rosa}
	\NCandidati{Autori}
	\Relatore{Dott.ssa Claudia Tomei}
	\NRelatore{Docente}{Docenti}
\end{frontespizio}

\end{document}

% !TEX encoding = UTF-8
% !TEX TS-program = pdflatex
% !TEX root = ../Articolo.tex
% !TEX spellcheck = it-IT

%*******************************************************
% Sommario+Abstract
%*******************************************************
\phantomsection
\pdfbookmark{Sommario}{Sommario}
\section*{Sommario}

Lo scopo di questo lavoro è fare un'analisi di rete usando un campione di antenne telefoniche cellulari, ipotizzando di voler costruire una ipotetica *mesh network* con esse. Abbiamo scelto di analizzare il sistema delle antenne comprese entro il Raccordo Anulare di Roma. Non sapendo a priori di quale tipo di rete si tratta, e che tipo di grafo si costruisce da essa, sono state studiate alcune proprietà topologiche e comportamentali dell'insieme di antenne studiate.
Dopo aver definito il criterio con cui dare un link a due nodi, sono state calcolate le matrici di adiacenza della rete complessiva e di quelle composte solo dalle antenne dei singoli gestori presenti in Italia. La prima parte dell'analisi consiste nell'estrapolare dai grafi ottenuti le distribuzioni dei gradi e nell'affidare all'istogramma dei gradi la forma funzionale che meglio gli si adatti. Avendo così ottenuto indicazioni significative sui tipi di rete in analisi, le abbiamo verficate studiando il comportamento percolativo della rete, mediante rimozione di nodi in due diversi scenari: un attacco intenzionale, che rimuovesse le antenne a partire da quelle con maggior grado, e una caduta random del sistema. I risultati ottenuti sono stati confrontati con modelli di rete esponenziali e scale-free.
L'esposizione è articolata nel seguente modo:

\begin{description}
\item[{\hyperref[sec:teoria]{Nella sezione \ref{sec:teoria}}}] verranno esposte le basi teoriche dello studio effettuato, con particolare accento ai modelli di rete utilizzati come riferimento e ai concetti di percolazione e soglia percolativa.

\item[{\hyperref[sec:dati]{Nella sezione \ref{sec:dati}}}] si spiega come sono stati raccolti e organizzati i dati, come è stata definita la rete e quindi con quali criteri è stata definita la matrice di adiacenza. Successivamente, accennando ai problemi computazionali avuti nel gestire la rete complessiva da circa 7000 nodi, verranno calcolate le matrici di adiacenza e le distribuzioni del grado.

\item[{\hyperref[sec:attaack]{Nella sezione \ref{sec:attaack}}}] viene effettuato lo studio percolativo mediante i due scenari di rimozione di nodi,  analizzando l'andamento del diametro e delle dimensioni del cluster più grande con la progressiva caduta delle antenne. I risultati sono stati confrontati con quelli ottenuti dai modelli, i quali hanno rivelato delle ambiguità. Nel caso di attacco random è stato testato infine un ipotetico effetto cascata causato dall'overload delle antenne rimaste (previa rimozione dei grossi hub che servono proprio a scongiurare tale evenienza).

\end{description}

